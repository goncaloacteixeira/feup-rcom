\addcontentsline{toc}{subsection}{Experiência 2 - Implementar duas LANs Virtuais no Switch}
\subsection*{Experiência 2 - Implementar duas LANs Virtuais no Switch}
O objetivo desta experiência é implementar duas VLANs num Cisco Switch, uma VLAN é uma rede virtual local.

\subsubsection{1. Como configurar a VLANy0?}
Primeiro é necessário ligar um cabo série do tux3 ao switch para aceder ao terminal de configuração (\emph{configure terminal}) do switch. De seguida cria-se uma vlan, de ID y0, no caso, 31. Por fim resta atribuir as portas em questão a essa vlan que acabou de ser criada.
\begin{lstlisting}[language=bash]
configure terminal
> vlan y0
> end
> configure terminal
> interface fastethernet 0/[n da porta]
> switchport mode access
> switchport access vlan y0
> end

\end{lstlisting}

\subsubsection{2. Quantos domínios de broadcast existem? O que se pode concluir a partir dos registos?}
O tux3 recebe resposta do tux4 quando faz \emph{ping broadcast}, mas não recebe do tux2. O tux2 não recebe nenhuma resposta quando executa a instrução de \emph{ping broadcast}. Desta forma pode-se concluir que existem dois domínios de broadcast, um que contem o tux3 e tux4, e outro que contém o tux2.
